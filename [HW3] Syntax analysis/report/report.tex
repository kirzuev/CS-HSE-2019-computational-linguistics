\documentclass[oneside,final,12pt]{article}

\usepackage[T2A]{fontenc}
%\usepackage[utf8]{inputenc}   % older versions of ucs package
\usepackage[utf8x]{inputenc}  % more recent versions (at least>=2004-17-10)
\usepackage[russian]{babel}
\usepackage{graphicx}
\usepackage{vmargin}
\setpapersize{A4}
\setmarginsrb{3cm}{2cm}{1.5cm}{2cm}{0pt}{0mm}{0pt}{13mm}
\usepackage{indentfirst}
\usepackage[footnotesize]{caption2}
\usepackage{alltt}

\sloppy

% Параметры страницы
\textheight=24cm
\textwidth=16cm
\footnotesep=3ex
\raggedbottom
\tolerance 3000
% подавить эффект "висячих стpок"
\clubpenalty=10000
\widowpenalty=10000
\renewcommand{\baselinestretch}{1.1}
\renewcommand{\baselinestretch}{1.5} %для печати с большим интервалом

\begin{document}

\begin{titlepage}
\begin{center}

    Национальный исследовательский университет\\
    Высшая школа экономики\\
    Факультет компьютерных наук\\[60mm]
    \bigskip
    Отчет по домашнему заданию №3 \\[5mm]   
    \textsf{\large\bfseries
        Синтаксический анализ, коллокации
    }\\[50mm]

   
    \begin{flushright}
        \parbox{0.4\textwidth}{
            студент 1 курса магистратуры\\
            \emph{Зуев Кирилл Александрович}\\[5mm]
        }
    \end{flushright}

    \vspace{\fill}
    Москва, 2019
\end{center}
\end{titlepage}

\newpage

\renewcommand{\contentsname}{Содержание}
\tableofcontents

\newpage

\section{Постановка задачи выбранного варианта (C)}

Составить (с использованием любого модуля морфоанализа) программу, выполняющую синтаксическую сегментацию текста на русском языке на базе выбранных правил:

\begin{itemize}
	\item сегментацию на простые предложения по знакам пунктуации \textbf{и/или}
	\item выделение неразрывных синтаксически связанных групп слов на основе локальных высоковероятных связей.
\end{itemize}

Протестировать программу на нескольких текстовых фрагментах (не менее 1-2 страниц).

\section{Уточнение постановки задачи}

В процессе выполнения задания я решил использовать два морфологических анализатора: \textbf{Mystem} (для токенизации текстов) и \textbf{PyMorphy} (для получения морфологических признаков каждого токена). \textbf{Mystem} тоже выдает морфологические признаки, но он снимает омонимию, оставляя только один вариант признаков, который в некоторых случаях оказывается неправильным и не позволяет корректно выполнить сегментацию. Поэтому я решил воспользоваться \textbf{PyMorphy}, так как он выдает все варианты разбора словоформы.\\

Мной выполнены оба пункта задания: и сегментация текста на простые предложения по знакам пунктуации, и выделение неразрывных синтаксически связанных групп слов. Для выделения групп использовались следующие высоковероятные связи (и их комбинация):

\begin{itemize}
	\item прилагательное/причастие и существительное, если у них совпадает падеж;
	\item краткое прилагательное/причастие и существительное;
	\item прилагательное/причастие и существительное, если существительное является неизменяемым;
	\item предлог и существительное;
	\item частица не/ни и прилагательное/причастие/существительное.
\end{itemize}

Сегментация на простые предложения выполнена примитивно: если встречаем знак препинания, то отделяем эту часть. Это очень простой способ, который порождает много ошибок и неточностей. Например, разделение однородных членов предложения, вводные слова, причастные и деепричастные обороты и др. Можно ориентироваться по союзам или членам потенциального простого предложения. В этом плане есть еще огромный простор для улучшения работы программы.\\

Для тестирования и демонстрации работы программы я взял тексты Тотального диктанта за 2015 -- 2018 гг.

\section{Пример вывода}

Ниже представлен пример обработанного абзаца текста:
\begin{verbatim}
[{Профессорская дача} {на берегу} {Финского залива}]. [{В отсутствие} хозяина],
[друга {моего отца}], [{нашей семье} позволялось там жить].
[Даже {спустя десятилетия} помню], [как {после утомительной дороги}
{из города} меня обволакивала прохлада {деревянного дома}],
[как собирала растрясшееся], [распавшееся {в экипаже} тело].
[{Эта прохлада} не была связана {со свежестью}], [скорее], [как ни странно],—
[{с упоительной затхлостью}], [в которой слились ароматы {старых книг} и
{многочисленных океанских трофеев}], [непонятно как
{доставшихся профессору-юристу}]. [Распространяя {солоноватый запах}],
[{на полках} лежали {засушенные морские звёзды}], [{перламутровые раковины}],
[{резные маски}], [{пробковый шлем} и даже игла рыбы-иглы].
\end{verbatim}

Полный разбор текстов можно найти в конечном выводе работы программы.

\section{Код программы}

Реализация программы находится в приложенных файлах \textbf{HW3.pdf} и \textbf{HW3.ipynb}.

\section{Выводы по исследованию}

В результате работы программы довольно хорошо выделяются именные группы с одним или несколькими эпитетами, допускающие наличие предлога и/или частицы не/ни в начале группы. В программе не хватает обработки глагольных групп, наречий и др. Кроме того, стоит строже разделять простые предложения, добавив дополнительные условия, так как обработка по знакам препинания допускает много случаев, не соответствующих верному разделению. Но в целом в результате получаются вполне цельные части, которые могут упростить синтаксический анализ большого текста.

\end{document}
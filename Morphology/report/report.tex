\documentclass[oneside,final,12pt]{article}

\usepackage[T2A]{fontenc}
%\usepackage[utf8]{inputenc}   % older versions of ucs package
\usepackage[utf8x]{inputenc}  % more recent versions (at least>=2004-17-10)
\usepackage[russian]{babel}
\usepackage{graphicx}
\usepackage{vmargin}
\setpapersize{A4}
\setmarginsrb{3cm}{2cm}{1.5cm}{2cm}{0pt}{0mm}{0pt}{13mm}
\usepackage{indentfirst}
\usepackage[footnotesize]{caption2}
\usepackage{alltt}

\sloppy

% Параметры страницы
\textheight=24cm
\textwidth=16cm
\footnotesep=3ex
\raggedbottom
\tolerance 3000
% подавить эффект "висячих стpок"
\clubpenalty=10000
\widowpenalty=10000
\renewcommand{\baselinestretch}{1.1}
\renewcommand{\baselinestretch}{1.5} %для печати с большим интервалом

\begin{document}

\begin{titlepage}
\begin{center}

    Национальный исследовательский университет\\
    Высшая школа экономики\\
    Факультет компьютерных наук\\[60mm]
    \bigskip
    Отчет по домашнему заданию №1 \\[5mm]   
    \textsf{\large\bfseries
        Графематический и морфологический анализ текста
    }\\[50mm]

   
    \begin{flushright}
        \parbox{0.4\textwidth}{
            студент 1 курса магистратуры\\
            \emph{Зуев Кирилл Александрович}\\[5mm]
        }
    \end{flushright}

    \vspace{\fill}
    Москва, 2019
\end{center}
\end{titlepage}

\newpage

\renewcommand{\contentsname}{Содержание}
\tableofcontents

\newpage

\section{Постановка задачи выбранного варианта (F)}

Провести исследование качества разрешения морфологической омонимии одного из 	морфоанализаторов для русского языка, подключив его к своей  программе.
Исследование можно провести вручную, взяв нескольких текстов небольшого размера (20 -- 25 предложений), либо автоматически, используя как эталон размеченные тексты. В последнем случае следует вычислить точность разрешения омонимии по леммам, части речи, а также по всем морфологическим характеристикам/тегам. Для исследования можно взять одну из следующих пар анализатор – размеченный текст:

\begin{itemize}
	\item mystem – тексты НКРЯ (RNC);
	\item pymorphy – OpenCorpora;
	\item CrossMorphy – OpenCorpora.
\end{itemize}

\section{Уточнение постановки задачи}

Мной было выбрано  \textbf{автоматическое} тестирование морфологического анализатора \textbf{pymorphy}, а в качестве эталона используются размеченные тексты со снятой омонимией из корпуса \textbf{OpenCorpora}.

\section{Определение точности разрешения разных типов омонимии}

Необходимо определить точность разрешения морфологическим анализатором разных типов омонимии:

\begin{itemize}
	\item по лемме;
	\item по части речи;
	\item по всем морфологическим характеристикам.
\end{itemize}

\textbf{pymorphy} при обработке входного слова выдает список всевозможных вариантов его разбора по морфологическим характеристикам. В данном случае используется бесконтекстный способ разрешения омонимии, так как каждому варианту сопоставлена некоторая вероятность. Таким образом, выбирается вариант разбора с наибольшей вероятностью, то есть первый из полученного списка.\\

Для проверки правильности разрешения омонимии по лемме необходимо сравнить полученную лемму из набора характеристик с леммой этого же слова из размеченного корпуса. Если они совпадают, то мы считаем разрешение омонимии верным для данного типа.\\

В случае разрешения омонимии по части речи полученная в ходе разбора анализатором часть речи с частью речи из эталонного корпуса. И если они совпадают, то считаем, что омонимии данного типа нет и разбор слова произошел верно.\\

Разрешение омонимии по всем морфологическим характеристикам предполагает, что все характеристики, включая лемму и часть речи, должны быть идентичными у полученного разбора и у эталона. Если все совпадает, то считаем для данного типа омонимии разбор верным.\\

В результате работы программы были получены следующие результаты точности:

\begin{itemize}
	\item точность разрешения омонимии по лемме: 0.92296;
	\item точность разрешения омонимии по части речи: 0.95848;
	\item точность разрешения омонимии по всем морфологическим характеристикам: 0.70885.
\end{itemize}

\section{Код программы}

Реализация программы находится в приложенных файлах \textbf{HW1.pdf} и \textbf{HW1.ipynb}.

\newpage


\section{Выводы по исследованию}

Бесконтекстное разрешение омонимии, встроенное в морфологический анализатор \textbf{pymorphy} с большой точностью определяет верную лемму и часть речи и с достаточно неплохой точностью, но хуже, определяет все морфологические характеристики целиком.

\end{document}